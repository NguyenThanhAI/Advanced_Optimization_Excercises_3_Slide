%% This Beamer template is based on the one found here: https://github.com/sanhacheong/stanford-beamer-presentation, and edited to be used for Stanford ARM Lab

\documentclass[10pt]{beamer}
%\mode<presentation>{}

\usepackage{media9}
\usepackage{amssymb,amsmath,amsthm,enumerate}
\usepackage{mathtools}
\usepackage[utf8]{inputenc}
\usepackage{array}
\usepackage[parfill]{parskip}
\usepackage[utf8]{vietnam}
\usepackage{graphicx,animate}
\usepackage{caption}
\usepackage{subcaption}
\usepackage{bm}
\usepackage{amsfonts,amscd}
\usepackage[]{units}
\usepackage{listings}
\usepackage{multicol}
\usepackage{multirow}
\usepackage{tcolorbox}
\usepackage{physics}
\usepackage{movie15}
% Enable colored hyperlinks
\hypersetup{colorlinks=true}

% The following three lines are for crossmarks & checkmarks
\usepackage{pifont}% http://ctan.org/pkg/pifont
\newcommand{\cmark}{\ding{51}}%
\newcommand{\xmark}{\ding{55}}%

% Numbered captions of tables, pictures, etc.
\setbeamertemplate{caption}[numbered]
\usepackage{media9} 
%\usepackage[superscript,biblabel]{cite}
%\usepackage{algorithmic}
%\usepackage{algorithm2e}
\usepackage{algpseudocode}
%\usepackage[linesnumbered,ruled,vlined]{algorithm2e}
\usepackage{algorithm}
%\usepackage{algorithmic}
\usepackage{caption}
%\usepackage{xcolor}
\usepackage{array}
%\renewcommand{\thealgocf}{}

\usepackage[natbib,backend=biber,style=ieee, sorting=ynt]{biblatex}
\bibliography{ref.bib}

\usepackage[acronym]{glossaries}

\usepackage{graphicx}
\graphicspath{{./figures}}
\usepackage{hyperref}

\theoremstyle{remark}
\newtheorem*{remark}{Remark}
\theoremstyle{definition}

\newcommand{\empy}[1]{{\color{darkorange}\emph{#1}}}
\newcommand{\empr}[1]{{\color{cardinalred}\emph{#1}}}
\newcommand{\examplebox}[2]{
\begin{tcolorbox}[colframe=darkcardinal,colback=boxgray,title=#1]
#2
\end{tcolorbox}}

\usetheme{Stanford} 
\input{./style_files_stanford/my_beamer_defs.sty}
\logo{\includegraphics[height=0.5in]{logos/HUS-name.jpg}}

\makeatletter
\let\@@magyar@captionfix\relax
\makeatother

\title[Tối ưu hóa nâng cao]{Tối ưu hóa nâng cao: Bài tập 3}

\begin{document}

\nocite{*}

\author[Nguyễn Chí Thanh - Nguyễn Đức Thịnh]{
	\begin{tabular}{c} 
	\Large
	Nguyễn Chí Thanh \\
    \footnotesize \href{mailto:nguyenchithanh\_sdh21@hus.edu.vn}{nguyenchithanh\_sdh21@hus.edu.vn} \\
	\Large
    Nguyễn Đức Thịnh \\
     \footnotesize \href{mailto:nguyenducthinh\_sdh21@hus.edu.vn}{nguyenducthinh\_sdh21@hus.edu.vn}
\end{tabular}
\vspace{-4ex}}

\institute{
	\vskip 5pt
	\begin{figure}
		\centering
		\begin{subfigure}[t]{0.5\textwidth}
			\centering
			\includegraphics[height=0.75in]{logos/HUS-logo.jpg}
		\end{subfigure}%
		~ 
		\begin{subfigure}[t]{0.5\textwidth}
			\centering
			\includegraphics[height=0.75in]{logos/MIM-logo.png}
		\end{subfigure}
	\end{figure}
	\vskip 10pt	
	Đại học Quốc Gia Hà Nội \\
	Trường đại học Khoa học tự nhiên\\
	Khoa Toán - Cơ - Tin học
	\vskip 3pt
}

\begin{noheadline}
\begin{frame} \maketitle \end{frame}
\end{noheadline}
    
\setbeamertemplate{itemize items}[default]
\setbeamertemplate{itemize subitem}[circle]

\begin{frame}{Nội dung}

    \begin{enumerate}
		\item Bài toán hồi quy tuyến tính
        \item So sánh ba thuật toán
        \item So sánh nhiều thuật toán
        \item Kết luận
    \end{enumerate}
    
\end{frame}

\begin{frame}[allowframebreaks]{Bài toán hồi quy tuyến tính}
	\begin{itemize}
		\item Hàm loss của Linear Regression:
		\begin{equation*}
			f(w) = \lVert X w - y \rVert_2^2
		\end{equation*}
		\item Ta đặt kích thước các ma trận $X \in \mathbb{R}^{m \times n}, y \in \mathbb{R}^m, w \in \mathbb{R}^n$, ta có $\mathrm{dom}(f)=\mathbb{R}^{n}$ là tập lồi
		\begin{equation*}
			\begin{aligned}
				f(w) &= (Xw - y)^T (Xw - y) \\
				&=w^T X^T X w - w^TX^Ty - y^T Xw + y^T y
			\end{aligned}
		\end{equation*}
		\item Do $w^TX^Ty$ và $y^TXw$ là hai số vô hướng và $(w^TX^Ty)^T=y^TXw$ nên $w^TX^Ty=y^TXw$ (chuyển vị của một số vô hướng bằng chính nó) nên:
		\begin{equation*}
			\begin{aligned}
				f(w) &= w^T X^T X w - w^TX^Ty - y^T Xw + y^T y\\
				&=w^T X^T X w-2w^TX^Ty+y^Ty
			\end{aligned}
		\end{equation*}
		\item Ta tính Gradient của $f(w)$ theo $w$:
		\begin{equation*}
			\nabla f(w) =\Big\lbrack X^TX + (X^TX)^T \Big\rbrack w - 2X^Ty = 2X^T X w - 2 X^T y
		\end{equation*}
		\item Ta tính ma trận Hessian của $f(w)$ theo $w$:
		\begin{equation*}
			\nabla^2 f(w) = 2X^T X
		\end{equation*}
		\item Ta xét:
		\begin{equation*}
			p^T \nabla^2 f(w) p = 2 p^T X^T X p = 2 (Xp)^TXp \geq 0 \thickspace\forall \thickspace p \in \mathbb{R}^{n}
		\end{equation*}
		nên:
		\begin{equation*}
			\nabla^2 f(w) \succeq 0 \thickspace \forall w \in \mathbb{R}^{n}
		\end{equation*}
		\item Vì $\mathrm{dom}(f)$ là tập lồi và $\nabla^2 f(w) \succeq 0$ nên hàm $f(w)$ là một hàm lồi.
	\end{itemize}
	Hàm loss được sử dụng trong lập trình:
	\begin{equation*}
		f(w) = \dfrac{1}{n}\sum_{i=1}^{n}\Big( x_i^T w - y_i \Big)^2
	\end{equation*}
	với $(x_i, y_i) \in \mathbb{R}^{n+1}, i=1, 2, \dots, n$ là cặp dữ liệu đầu vào và nhãn tương ứng với $w \in \mathbb{R}^n$
	
	\framebreak
	Nghiệm tối ưu (khi các cột của ma trận $X$ độc lập tuyến tính):

	\begin{equation*}
		w^{*} = \Big(X^T X \Big)^{-1}X^T y
	\end{equation*}

	Với dữ liệu tập train:

	\begin{equation*}
		w_{train}^{*} = \begin{bmatrix} -1.3962 &  0.3796 & 0.5107 & -0.04332 &\\ 0.07553 & 0.5786 & 0.1703 & 0.07757 &\\ 1.2154  & 1.2375 &  0.8117  & 0.16389 &\\ -0.5221 &  0.4350 & -0.0788 & -0.5564 \end{bmatrix}
	\end{equation*}

	và giá trị hàm loss tương ứng:

	\begin{equation*}
		f(w_{train}^{*}, \text{ data }_{\text{train}}) \approx 0.04804
	\end{equation*}

	\begin{equation*}
		f(w_{train}^{*}, \text{ data }_{\text{val}}) \approx 0.04959
	\end{equation*}

\end{frame}

\begin{frame}{So sánh ba thuật toán}
	\framesubtitle{GD: Fixed step sizes}
	\begin{figure}[h!]
		\centering
		\includegraphics[width=10cm]{Thinh/1.png}
	  \end{figure}

\end{frame}
\begin{frame}{So sánh ba thuật toán}
	\framesubtitle{GD: Fixed step sizes}

	  \begin{figure}[h!]
		\centering
		\includegraphics[width=9cm]{Thinh/2.png}
	  \end{figure}

\end{frame}
\begin{frame}{So sánh ba thuật toán}
	\framesubtitle{GD: Fixed step sizes}

	  \begin{figure}[h!]
		\centering
		\includegraphics[width=11cm]{Thinh/3.png}
	  \end{figure}


\end{frame}
\begin{frame}{So sánh ba thuật toán}
	\framesubtitle{GD: Fixed step sizes}

	  \begin{figure}[h!]
		\centering
		\includegraphics[width=11cm]{Thinh/4.png}
	  \end{figure}
	  Kết luận: Lựa chọn fixed step size = 0.1
\end{frame}

\begin{frame}{So sánh ba thuật toán}
	\framesubtitle{GD: Backtracking}
	\begin{figure}[h!]
		\centering
		\includegraphics[width=11cm]{Thinh/5.png}
	\end{figure}

\end{frame}

\begin{frame}{So sánh ba thuật toán}
	\framesubtitle{GD: Backtracking}
	\begin{figure}[h!]
		\centering
		\includegraphics[width=11cm]{Thinh/6.png}
	\end{figure}
	Kết luận: Lựa chọn initial = 10
\end{frame}



\begin{frame}{So sánh ba thuật toán}
	\framesubtitle{GD: Inverse decay}
	\begin{figure}[h!]
		\centering
		\includegraphics[width=11cm]{Thinh/7.png}
	\end{figure}
\end{frame}

\begin{frame}{So sánh ba thuật toán}
	\framesubtitle{GD: Inverse decay}

	\begin{figure}[h!]
		\centering
		\includegraphics[width=11cm]{Thinh/8.png}
	\end{figure}
\end{frame}

\begin{frame}{So sánh ba thuật toán}
	\framesubtitle{GD: Fixed vs. backtracking vs. inverse decay step sizes}
	\begin{figure}[h!]
		\centering
		\includegraphics[width=10cm]{Thinh/9.png}
	\end{figure}

\end{frame}

\begin{frame}{So sánh ba thuật toán}
	\framesubtitle{GD: Fixed vs. backtracking vs. inverse decay step sizes}

	\begin{figure}[h!]
		\centering
		\includegraphics[width=11cm]{Thinh/10.png}
	\end{figure}

	\begin{itemize}
		\item Fixed step size (size = 0.1)
		\item Backtracking (initial = 10)
		\item Inverse decay (initial = 1)
	\end{itemize}

\end{frame}
\begin{frame}{So sánh ba thuật toán}
	\framesubtitle{GD: Fixed vs. backtracking vs. inverse decay step sizes}

	\begin{figure}[h!]
		\centering
		\includegraphics[width=11cm]{Thinh/11.png}
	\end{figure}

	\begin{small}
	\begin{multicols}{2}
	\begin{itemize}
		\item Fixed step size (0.1): 0.3368s (501 steps) 
		\item Backtracking (10): 0.1560s (65 outer steps, 321 steps total)
		\item Inverse decay (1): không thể đạt được loss 0.05 (min loss = 0.0913)
		\item Fixed step size (0.1): 1.4577s (2175 steps)
		\item Backtracking (10): 0.6931s (305 outer steps, 1522 steps total)
		\item Inverse decay (1): không thể đạt được norm grad 1e-4 (min = 0.0037)
	\end{itemize}
	\end{multicols}
	\end{small}
	
\end{frame}

\begin{frame}{So sánh ba thuật toán}
	\framesubtitle{GD: Fixed vs. backtracking vs. inverse decay step sizes}
	\begin{itemize}
		\item Fixed step size (size = 0.1)
		\item Backtracking (initial = 10)
		\item Inverse decay (initial = 1)

	\end{itemize}
	Kết luận:
	\begin{itemize}
		\item Invese decay (1): không hội tụ
		\item Backtracking (10) có tổng số bước nhỏ hơn và hội tụ nhanh hơn so với fixed step size (size = 0.1)
		GD: lựa chọn backtracking (10) để so sánh với các thuật toán khác.
	\end{itemize}
\end{frame}

\begin{frame}{So sánh ba thuật toán}
	\framesubtitle{Newton: Fixed step sizes}
	\begin{itemize}
		\item Step size 0.0001: không hội tụ
		\item Step size 0.001: 10.6940s (7157 steps)
		\item Step size 0.01: 1.0869s (713 steps)
		\item Step size 0.1: 0.1060s (68 steps)
		\item Step size 1: 0.001s (1 step)
		\item Step size 2: không hội tụ
		\item Step size 5: không hội tụ
		\item Step size 10: không hội tụ

	\end{itemize}
	Chọn fixed step size = 1 
\end{frame}

\begin{frame}{So sánh ba thuật toán}
	\framesubtitle{Newton: Backtracking}
	\begin{figure}[h!]
		\centering
		\includegraphics[width=9cm]{Thinh/12.png}
	\end{figure}
	Kết luận: Lựa chọn initial = 10
\end{frame}

\begin{frame}{So sánh ba thuật toán}
	\framesubtitle{Newton: Fixed step size vs Backtracking}
	\centering
	\begin{table}[]
		\resizebox{\linewidth}{!}{
		\begin{tabular}{|| c| c| c||}
		\hline
		   Schedule                 & $\text{loss} \leq 0.05$                & $\text{grad norm} \leq 10^{-4}$      \\ \hline
		Fixed step size (1) & 0.001 (1 step)                        & 0.001 (1 step)                        \\ \hline
		Backtracking (10)   & 0.007 (2 outer steps, 4 steps total ) & 0.019 (6 outer steps, 20 steps total) \\ [1ex]
		\hline
		\end{tabular}
		}
	\end{table}
	Newton: lựa chọn fixed step size (1) để so sánh với các thuật toán khác.

\end{frame}


\begin{frame}{So sánh ba thuật toán}
	\framesubtitle{Accelerated gradient: Fixed step sizes}

	\begin{figure}[h!]
		\centering
		\includegraphics[width=11cm]{Thinh/13.png}
	\end{figure}

\end{frame}
\begin{frame}{So sánh ba thuật toán}
	\framesubtitle{Accelerated gradient: Fixed step sizes}

	\begin{figure}[h!]
		\centering
		\includegraphics[width=9cm]{Thinh/14.png}
	\end{figure}

\end{frame}
\begin{frame}{So sánh ba thuật toán}
	\framesubtitle{Accelerated gradient: Fixed step sizes}

	\begin{figure}[h!]
		\centering
		\includegraphics[width=11cm]{Thinh/15.png}
	\end{figure}

\end{frame}
\begin{frame}{So sánh ba thuật toán}
	\framesubtitle{Accelerated gradient: Fixed step sizes}

	\begin{figure}[h!]
		\centering
		\includegraphics[width=11cm]{Thinh/16.png}
	\end{figure}

\end{frame}
\begin{frame}{So sánh ba thuật toán}
	\framesubtitle{Accelerated gradient: Fixed step sizes}

	\begin{figure}[h!]
		\centering
		\includegraphics[width=11cm]{Thinh/17.png}
	\end{figure}
	Kết luận: Lựa chọn fixed step size = 0.1

\end{frame}

\begin{frame}{So sánh ba thuật toán}
	\framesubtitle{Accelerated gradient: Backtracking}
	\begin{figure}[h!]
		\centering
		\includegraphics[width=11cm]{Thinh/18.png}
	\end{figure}
	Kết luận: Lựa chọn initial = 10
\end{frame}

\begin{frame}{So sánh ba thuật toán}
	\framesubtitle{Accelerated gradient: Inverse decay}
	\begin{figure}[h!]
		\centering
		\includegraphics[width=11cm]{Thinh/19.png}
	\end{figure}
	20 bước đầu tiên và 1000 bước đầu tiên
\end{frame}

\begin{frame}{So sánh ba thuật toán}
	\framesubtitle{Accelerated gradient: Inverse decay}
	\begin{figure}[h!]
		\centering
		\includegraphics[width=11cm]{Thinh/20.png}
	\end{figure}
	Bước 100-1000 và 1000-10000 => Loại 0.001, 0.001
\end{frame}

\begin{frame}{So sánh ba thuật toán}
	\framesubtitle{Accelerated gradient: Inverse decay}

	\begin{figure}[h!]
		\centering
		\includegraphics[width=11cm]{Thinh/21.png}
	\end{figure}
	Bước 100-1000 và 1000-10000 => Loại 0.1
\end{frame}

\begin{frame}{So sánh ba thuật toán}
	\framesubtitle{Accelerated gradient: Inverse decay}

	\begin{figure}[h!]
		\centering
		\includegraphics[width=11cm]{Thinh/22.png}
	\end{figure}
	Kết luận: chọn initial = 2
\end{frame}

\begin{frame}{So sánh ba thuật toán}
	\framesubtitle{Accelerated gradient: Fixed vs. backtracking vs. inverse decay step sizes }
	\begin{figure}[h!]
		\centering
		\includegraphics[width=6cm]{Thinh/23.png}
	\end{figure}
	\begin{itemize}
		\item Fixed step size (size = 0.1)
		\item Backtracking (initial = 10)
		\item Inverse decay (initial = 2)

	\end{itemize}
\end{frame}

\begin{frame}{So sánh ba thuật toán}
	\framesubtitle{Accelerated gradient: Fixed vs. backtracking vs. inverse decay step sizes }
	
\begin{table}[]
	\centering
	\resizebox{\linewidth}{!}{
	\begin{tabular}{|| c| c| c||}
	\hline
						& $\text{loss} \leq 0.05$                    & $\text{grad norm } \leq 10^{-4}  $        \\ \hline
	Fixed step size (1) & 0.0760s (55 steps)                        & 0.2309s (173 steps)                       \\ \hline
	Backtracking (10)   & 0.0830s (28 outer steps, 157 steps total) & 0.2720s (91 outer steps, 524 steps total) \\ \hline
	Inverse decay (2)   & 0.2020s (151 steps)                       & 0.7030s (529 steps)   \\     [1ex]   
	\hline           
	\end{tabular}
	}
	\end{table}

	Accelerated Gradient: lựa chọn fixed step size (0.1) để so sánh với các thuật toán khác
\end{frame}

\begin{frame}{So sánh ba thuật toán}
	\framesubtitle{GD vs. Accelerated gradient vs. Newton}
	\begin{figure}[h!]
		\centering
		\includegraphics[width=10cm]{Thinh/24.png}
	\end{figure}
\end{frame}

\begin{frame}{So sánh ba thuật toán}
	\framesubtitle{Accelerated gradient: Fixed vs. backtracking vs. inverse decay step sizes }
	\begin{itemize}
		\item GD (backtracking initial = 10)
		\item Newton (fixed step size = 1)
		\item Accelerated gradient (fixed step size = 0.1)

	\end{itemize}

	\begin{table}[]
		\centering
		\resizebox{\linewidth}{!}{
		\begin{tabular}{|| c| c| c||}
		\hline
							 & $\text{loss} \leq 0.05$                    & $\text{grad norm } \leq 10^{-4}  $            \\ \hline
		GD                   & 0.1560s (65 outer steps, 321 steps total) & 0.6931s (305 outer steps, 1522 steps total) \\ \hline
		Newton               & 0.001s (1 step)                           & 0.001s (1 step)                             \\ \hline
		Accelerated Gradient & 0.0760s (55 steps)                        & 0.2309s (173 steps)         \\     [1ex] 
		\hline               
		\end{tabular}
		}
	\end{table}

	Với bài toán này, thuật toán Newton hội tụ nhanh nhất, sau đó là Accelerated gradient và thuật toán GD hội tụ chậm nhất
\end{frame}


\begin{frame}{So sánh ba thuật toán}
	\framesubtitle{GD vs. Accelerated gradient vs. Newton}
	\begin{figure}[h!]
		\centering
		\includegraphics[width=10cm]{Thinh/25.png}
	\end{figure}
\end{frame}

\begin{frame}{So sánh ba thuật toán}
	\framesubtitle{GD vs. Accelerated gradient vs. Newton}
	\begin{figure}[h!]
		\centering
		\includegraphics[width=10cm]{Thinh/26.png}
	\end{figure}
\end{frame}

\begin{frame}{So sánh ba thuật toán}
	\framesubtitle{GD vs. Accelerated gradient vs. Newton}
	\begin{figure}[h!]
		\centering
		\includegraphics[width=10cm]{Thinh/27.png}
	\end{figure}
\end{frame}

\begin{frame}{So sánh ba thuật toán}
	\framesubtitle{GD vs. Accelerated gradient vs. Newton}
	\begin{figure}[h!]
		\centering
		\includegraphics[width=10cm]{Thinh/28.png}
	\end{figure}
\end{frame}

\begin{frame}{So sánh nhiều thuật toán}
	Các thuật toán được so sánh:
	\begin{multicols}{3}
	\begin{enumerate}
		\item Gradient Descent
  		\item Newton
    	\item Accelerated
     	\item BFGS
      	\item DFP
        \item Momentum
        \item Adagrad
        \item Adadelta
        \item RMSProp
        \item Adam
        \item Adamax
        \item AMSGrad
        \item Nadam
        \item AdaBelief
        \item RAdam
        \item Avagrad    
	\end{enumerate}
	\end{multicols}
	Các step size: $10^{-4}, 10^{-3}, 10^{-2}, 10^{-1}, 1, 2, 5, 10$
	
	Các kiểu chọn step size từng bước: Fixed, Backtracking, Inverse decay, Warmup, Triangular Cyclic

\end{frame}
	%\begin{figure}
	%	\centering
	%	\begin{subfigure}[b]{0.45\textwidth}
	%		\centering
	%		\includegraphics[width=\textwidth]{Thanh/warmup_lr.png}
	%		\caption{$y=x$}
	%	\end{subfigure}
	%	\hfill
	%	\begin{subfigure}[b]{0.45\textwidth}
	%		\centering
	%		\includegraphics[width=\textwidth]{Thanh/triangular_cyclic.png}
	%		\caption{$y=3sinx$}
	%	\end{subfigure}
	%\end{figure}

\begin{frame}{So sánh nhiều thuật toán}
	\begin{figure}[!htp]
		% Maximum length
		\subfloat[Warmup step size]{\label{fig1:a}\includegraphics[width=0.5\linewidth]{Thanh/warmup_lr.png}}\hfill
		\subfloat[Triangular cyclic step size]{\label{fig1:b}\includegraphics[width=0.5\linewidth]{Thanh/triangular_cyclic.png}}%
	  \end{figure}

	Mỗi nhóm gồm thuật toán tối ưu, step size, kiểu chọn step size sẽ tạo thành một tổ hợp.
	Một tổ hợp được dùng để train 5000 bước

\end{frame}

\begin{frame}[allowframebreaks]{So sánh nhiều thuật toán}
	\framesubtitle{Các hình vẽ so sánh với trục hoành là số bước}
	\begin{figure}
		\centering
		\includegraphics[width=10cm]{Thanh/backtrack-lr-step-op_step.png}
		\caption{So sánh các thuật toán với phương pháp backtracking tìm step size}
	\end{figure}

	\begin{figure}
		\centering
		\includegraphics[width=10cm]{Thanh/cyclic-lr-step-op_step.png}
		\caption{So sánh các thuật toán với phương pháp cyclic}
	\end{figure}

	\begin{figure}
		\centering
		\includegraphics[width=10cm]{Thanh/fixed-lr-step-op_step.png}
		\caption{So sánh các thuật toán với phương pháp cố định step size}
	\end{figure}

	\begin{figure}
		\centering
		\includegraphics[width=10cm]{Thanh/warmup-lr-step-op_step.png}
		\caption{So sánh các thuật toán với phương pháp warmup step size}
	\end{figure}

	\begin{figure}
		\centering
		\includegraphics[width=10cm]{Thanh/inverse-lr-step-op_step.png}
		\caption{So sánh các thuật toán với phương pháp giảm step size dạng hyperbol}
	\end{figure}
\end{frame}


\begin{frame}[allowframebreaks]{So sánh nhiều thuật toán}
	\framesubtitle{Các hình vẽ so sánh với trục hoành là thời gian (giây)}
	\begin{figure}
		\centering
		\includegraphics[width=10cm]{Thanh/backtrack-lr-step-op_time.png}
		\caption{So sánh các thuật toán với phương pháp backtracking tìm step size}
	\end{figure}

	\begin{figure}
		\centering
		\includegraphics[width=10cm]{Thanh/cyclic-lr-step-op_time.png}
		\caption{So sánh các thuật toán với phương pháp cyclic}
	\end{figure}

	\begin{figure}
		\centering
		\includegraphics[width=10cm]{Thanh/fixed-lr-step-op_time.png}
		\caption{So sánh các thuật toán với phương pháp cố định step size}
	\end{figure}

	\begin{figure}
		\centering
		\includegraphics[width=10cm]{Thanh/warmup-lr-step-op_time.png}
		\caption{So sánh các thuật toán với phương pháp warmup step size}
	\end{figure}

	\begin{figure}
		\centering
		\includegraphics[width=10cm]{Thanh/inverse-lr-step-op_time.png}
		\caption{So sánh các thuật toán với phương pháp giảm step size dạng hyperbol}
	\end{figure}
\end{frame}

\begin{frame}{So sánh nhiều thuật toán}

	Thời gian đạt điều kiện hội tụ nhỏ nhất của từng thuật toán:
	

	Tập train (sự thay đổi loss tương đối nhỏ hơn $10^{-3}$ và chuẩn của vector gradient nhỏ hơn $10^{-4}$)
	Tập train (sự thay đổi loss tương đối nhỏ hơn $10^{-3}$ và chuẩn của vector gradient nhỏ hơn $10^{-3}$)

\end{frame}

\begin{frame}{So sánh nhiều thuật toán}

	\begin{table} [h!]
		\centering
		\resizebox{\linewidth}{!}{
		\begin{tabular}{ || c | c | c | c | c | c | c | c | c | c|| }
		\hline
		Thuật toán & Steps & Schedule & Step size & Loss & MAE & Inner steps & Time (s) & Grad norm & $R^2$ \\ [0.5 ex]
		\hline \hline
		GD & 304 & backtracking & 5 & 0.04698 & 0.1396 & 1218 & 0.7623 & 9.7491e-05 & 0.6548 \\ \hline
		Newton & 1 &  backtracking &  1 &  0.04642&  0.1396 & 1 &  0.004000&  1.7353e-16&  0.6589  \\ \hline
		Accelerated & 90 &  backtracking &  5 &  0.04678 &  0.1396 &  434 &  0.2740 &  8.9588e-05 &  0.6562 \\ \hline
		BFGS &28 &  backtracking &  10 &  0.04675 &  0.1399 &  82 &  0.07697 &  6.7362e-05 &  0.6565 \\ \hline
		DFP & 42 &  backtracking &  10 &  0.04690 &  0.1395 &  109 &  0.1965 &  9.5577e-05 &  0.6554 \\ \hline
		Adam & 55 &  backtracking &  5 &  0.04649 &  0.1394 &  383 &  0.1470 &  8.5136e-05 &  0.6584\\ \hline
		Avagrad & Null \\ \hline 
		RAdam &64 &  backtracking &  2 &  0.04643 &  0.1394 &  288 &  0.1457 &  5.9528e-05 &  0.6588 \\ \hline
		Momentum &  83 &  backtracking &  5 &  0.04685 &  0.1401 &  495 &  0.2240 &  9.9035e-05 &  0.6557 \\ \hline
		Adagrad & 223 &  backtracking &  2 &  0.0465 &  0.1392 &  629 &  0.5130 &  9.5012e-05 &  0.6579 \\ \hline
		RMSProp & 301 &  backtracking &  5 &  0.04644 &  0.1400 &  3130 &  0.9421 &  8.6495e-05 &  0.6588\\ \hline
		Adadelta & 525 &  backtracking &  2 &  0.04654 &  0.1398 &  12219 &  2.7243 &  9.9677e-05 &  0.6581\\ \hline
		Adamax & 65 &  backtracking &  5 &  0.04675 &  0.1403 &  413 &  0.1679 &  9.3636e-05 &  0.6565 \\ \hline
		Nadam & 77 &  backtracking &  5 &  0.04655 &  0.1394 &  611 &  0.2230 &  8.5295e-05 &  0.6580 \\ \hline
		AMSGrad & 76 &  backtracking &  10 &  0.04650 &  0.1393 &  586 &  0.2758 &  7.6637e-05 &  0.6583 \\ \hline
		AdaBelief & 41 &  backtracking &  5 &  0.04647 &  0.1393 &  294 &  0.1180 &  8.4853e-05 &  0.6585 \\ [1ex]
		\hline
		\end{tabular}
		}
		\caption{So sánh thời gian đạt điều kiện hội tụ trên tập train}
	\end{table}

\end{frame}

\begin{frame}{So sánh nhiều thuật toán}
	\begin{table} [h!]
		\centering
		\resizebox{\linewidth}{!}{
		\begin{tabular}{ || c | c | c | c | c | c | c | c | c | c|| }
		\hline
		Thuật toán & Steps & Schedule & Step size & Loss & MAE & Inner steps & Time (s) & Grad norm & $R^2$ \\ [0.5 ex]
		\hline \hline
		GD & 127 &  backtracking &  5 &  0.04840 &  0.1409 &  511 &  0.3030 &  0.0003 &  0.6282 \\ \hline
		Newton & 1 &  backtracking &  1 &  0.04750 &  0.1413 &  1 &  0.0040 &  0.0007 &  0.6351  \\ \hline
		Accelerated & 37 &  backtracking &  5 &  0.04891 &  0.1437 &  179 &  0.1119 &  0.0007 &  0.6243 \\ \hline
		BFGS &11 &  backtracking &  10 &  0.0492 &  0.1437 &  35 &  0.02998 &  0.0006 &  0.6214 \\ \hline
		DFP & 15 &  backtracking &  10 &  0.04915 &  0.1443 &  39 &  0.09148 &  0.0006 &  0.6225 \\ \hline
		Adam & 26 &  backtracking &  5 &  0.04797 &  0.1431 &  130 &  0.06000 &  0.0007 &  0.6315\\ \hline
		Avagrad & 129 &  fixed &  10 &  0.0689 &  0.1707 &  129 &  0.1146 &  0.0009 &  0.4704 \\ \hline 
		RAdam &31 &  backtracking &  1 &  0.04846 &  0.1421 &  66 &  0.06105 &  0.0003 &  0.6277 \\ \hline
		Momentum &  33 &  backtracking &  5 &  0.05161 &  0.1529 &  428 &  0.1320 &  0.0009 &  0.6036 \\ \hline
		Adagrad & 86 &  backtracking &  2 &  0.04813 &  0.1420 &  246 &  0.1910 &  0.0009 &  0.6303 \\ \hline
		RMSProp & 301 &  backtracking &  5 &  0.04761 &  0.1417 &  3130 &  0.9421 &  0.0006 &  0.6343\\ \hline
		Adadelta & 85 &  backtracking &  5 &  0.04833 &  0.1429 &  330 &  0.1920 &  0.0009 &  0.6288\\ \hline
		Adamax &29 &  backtracking &  1 &  0.05057 &  0.1492 &  319 &  0.1200 &  0.0009 &  0.6116 \\ \hline
		Nadam & 32 &  backtracking &  2 &  0.04873 &  0.1449 &  147 &  0.07400 &  0.0009 &  0.6257 \\ \hline
		AMSGrad & 20 &  backtracking &  5 &  0.05018 &  0.1434 &  150 &  0.057002 &  0.0006 &  0.6146 \\ \hline
		AdaBelief & 23 &  backtracking &  5 &  0.04848 &  0.1422 &  129 &  0.06500 &  0.0003 &  0.6276 \\ [1ex]
		\hline
		\end{tabular}
		}
		\caption{So sánh thời gian đạt điều kiện hội tụ trên tập validation}
	\end{table}

\end{frame}

\begin{frame}{So sánh nhiều thuật toán}

	\begin{table} [h!]
		\centering
		\resizebox{\linewidth}{!}{
		\begin{tabular}{ || c | c | c | c | c | c | c | c | c | c | c|| }
		\hline
		Thuật toán & Min loss & Schedule & Step size & MAE & Steps & Inner steps & Time (s) & Grad norm & $R^2$ & Solution distance \\ [0.5 ex]
		\hline \hline
		GD & 0.04642 &  backtracking &  5 &  0.1395 &  4999 &  19998 &  11.9320 &  5.9560e-06 &  0.6589 & 0.03463 \\ \hline
		Newton & 0.04642 &  inverse decay &  2 &  0.1395 &  4285 &  4285 &  6.4144 &  1.1830e-05 &  0.6589 & 12248 \\ \hline
		Accelerated & 0.04642 &  backtracking &  5 &  0.1396 &  4855 &  23379 &  16.02871 &  2.7890e-09 &  0.6589 & 0.03440 \\ \hline
		BFGS &0.04642 &  warmup &  1 &  0.1396 &  3708 &  3708 &  6.1286 &  6.8517e-14 &  0.6589 & 0.03440 \\ \hline
		DFP & 0.04642 &  warmup &  1 &  0.1396 &  4495 &  4495 &  6.3558 &  2.6095e-12 &  0.6589 & 0.03440 \\ \hline
		Adam & 0.04642 &  backtracking &  0.001 &  0.1396 &  3664 &  33971 &  12.6386 &  5.4339e-11 &  0.6589 & 0.04408\\ \hline
		Avagrad & 0.04972 &  backtracking &  10 &  0.1411 &  4999 &  5055 &  9.5528 &  0.0008 &  0.6347 & 0.07974 \\ \hline 
		RAdam &0.04642 &  backtracking &  0.001 &  0.1396 &  3987 &  41786 &  14.8537 &  8.07073e-11 &  0.6589 & 0.04407 \\ \hline
		Momentum &  0.04642 &  backtracking &  10 &  0.1396 &  4612 &  17849 &  9.9354 &  4.5472e-11 &  0.6589 & 0.03440 \\ \hline
		Adagrad & 0.04642 &  backtracking &  1 &  0.1396 &  3986 &  8605 &  8.1345 &  1.2800e-10 &  0.6589 & 0.03861 \\ \hline
		RMSProp & 0.04642 &  backtracking &  0.1 &  0.1396 &  4874 &  97153 &  26.7683 &  1.9558e-12 &  0.6589  & 0.03921\\ \hline
		Adadelta & 0.04642 &  backtracking &  2 &  0.1396 &  4999 &  1231808 &  218.7172 &  1.6255e-05 &  0.6589 & 0.02898\\ \hline
		Adamax &0.04642 &  backtracking &  0.01 &  0.13961 &  4104 &  75850 &  18.7867 &  2.2710e-11 &  0.6589 & 0.03440 \\ \hline
		Nadam & 0.04642 &  backtracking &  0.01 &  0.1396 &  4882 &  107711 &  24.3817 &  5.6385e-11 &  0.6589 & 0.04182 \\ \hline
		AMSGrad & 0.04642 &  warmup &  2 &  0.1396 &  4452 &  4452 &  3.5742 &  1.7151e-12 &  0.6589 & 0.03856 \\ \hline
		AdaBelief & 0.04642 &  backtracking &  0.0001 &  0.1396 &  1794 &  19980 &  6.0632 &  3.2365e-11 &  0.6589 & 0.03994 \\ [1ex]
		\hline
		\end{tabular}
		}
		\caption{So sánh thời gian đạt giá trị loss nhỏ nhất trên tập train}
	\end{table}

\end{frame}


\begin{frame}{So sánh nhiều thuật toán}

	\begin{table} [h!]
		\centering
		\resizebox{\linewidth}{!}{
		\begin{tabular}{ || c | c | c | c | c | c | c | c | c | c|| }
		\hline
		Thuật toán & Min loss & Schedule & Step size & MAE & Steps & Inner steps & Time (s) & Grad norm & $R^2$ \\ [0.5 ex]
		\hline \hline
		GD & 0.04720 &  cyclic &  1 &  0.1400 &  1778 &  1778 &  1.2055 &  0.0001 &  0.6374 \\ \hline
		Newton & 0.04733 &  fixed &  0.01 &  0.14063 &  1436 &  1436 &  2.2674 &  0.0009 &  0.6364  \\ \hline
		Accelerated & 0.04727 &  cyclic &  0.01 &  0.1405 &  1371 &  1371 &  1.9054 &  0.0007 &  0.6369\\ \hline
		BFGS &0.04729 &  warmup &  0.1 &  0.1405 &  328 &  328 &  0.4802 &  0.0002 &  0.6368 \\ \hline
		DFP & 0.04727 &  warmup &  1 &  0.1391 &  446 &  446 &  0.6439 &  0.0006 &  0.6369 \\ \hline
		Adam & 0.04730 &  fixed &  0.001 &  0.1409 &  1073 &  1073 &  0.7792 &  0.0004 &  0.6367\\ \hline
		Avagrad & 0.04958 &  backtracking &  10 &  0.1428 &  4999 &  5055 &  9.5528 &  0.001417 &  0.6192 \\ \hline 
		RAdam &0.04731 &  backtracking &  0.001 &  0.1410 &  1095 &  1095 &  2.2251 &  0.0005 &  0.6366 \\ \hline
		Momentum & 0.04732 &  backtracking &  10 &  0.1409 &  140 &  634 &  0.3140 &  0.0004 &  0.6365 \\ \hline
		Adagrad & 0.04735 &  fixed &  2 &  0.1410 &  648 &  648 &  0.5333 &  0.0007 &  0.6363 \\ \hline
		RMSProp & 0.04730 &  inverse decay &  1 &  0.1408 &  607 &  607 &  0.4155 &  0.0002 &  0.6367\\ \hline
		Adadelta & 0.04731 &  cyclic &  1 &  0.1408 &  4467 &  4467 &  3.0681 &  0.0002 &  0.6366\\ \hline
		Adamax &0.04732 &  backtracking &  5 &  0.1409 &  115 &  660 &  0.2840 &  0.0005 &  0.6365 \\ \hline
		Nadam & 0.04731 &  backtracking &  0.001 &  0.1409 &  1169 &  1169 &  1.9850 &  0.0005 &  0.6366 \\ \hline
		AMSGrad & 0.04734 &  fixed &  2 &  0.1411 &  468 &  468 &  0.3943 &  0.0007 &  0.6364 \\ \hline
		AdaBelief & 0.04735 &  inverse decay &  1 &  0.1406 &  203 &  203 &  0.1407 &  0.0002 &  0.6363 \\ [1ex]
		\hline
		\end{tabular}
		}
		\caption{So sánh thời gian đạt giá trị loss nhỏ nhất trên tập validation}
	\end{table}

\end{frame}

\begin{frame}{Kết luận}
	\begin{itemize}
		\item Thuật toán Newton cho tốc độ hội tụ nhanh
  		\item Phương pháp backtracking tìm step size cho tốc độ hội tụ rất tốt
    	\item Các phương pháp cyclic, warmup lại cho kết quả khá tốt trên tập validation (dù tốc độ hội tụ hoặc giảm không tốt cho tập train)
     	\item Khoảng thời gian đầu, tốc độ giảm của Accelerated Gradient nhanh hơn hẳn với GD dù chi phí tính toán không khác biết nhiều
	\end{itemize}
\end{frame}

\begin{frame}[allowframebreaks]{Tài liệu tham khảo}
    \printbibliography
\end{frame}

\begin{frame}[allowframebreaks]{Phụ lục}
	\begin{enumerate}
		\item  Gradient Descent
		\begin{equation*}
			\begin{cases}g_t = \nabla f(w_{t-1}) \\ w_t = w_{t-1} - \alpha g_t\end{cases}
		\end{equation*}
		\item Momentum
		\begin{equation*}
			\begin{cases}m_0 = 0 \\ g_t = \nabla f(w_{t-1}) 
				\\ m_t = \beta_1 m_{t-1} + (1-\beta_1)g_t\\ w_t = w_{t-1} - \alpha m_t\end{cases}
		\end{equation*}
		\item Adagrad
		\begin{equation*}
			\begin{cases}v_0 = 0\\ g_t = \nabla f(w_{t-1}) \\ v_t = v_{t-1} + g_t^2 \\ w_t = w_{t-1} - \alpha \dfrac{g_t}{\sqrt{v_t} + \epsilon} \end{cases}
		\end{equation*}
		\item Adadelta
		\begin{equation*}
			\begin{cases}v_0 = 0\\d_0 = 0 \\ g_t = \nabla f(w_{t-1}) \\ v_t = \beta v_{t-1} + (1-\beta)g_t^2 \\ \Delta w = - \alpha \dfrac{\sqrt{d_{t-1} + \epsilon}g_t}{\sqrt{v_t + \epsilon}} \\ w_t = w_{t-1} + \Delta w \\ d_t = \beta d_{t-1} + (1 - \beta) \Delta w^2 \end{cases}
		\end{equation*}
		\item RMSProp
		\begin{equation*}
			\begin{cases}v_0 = 0\\ g_t = \nabla f(w_{t-1}) \\ v_t = \beta v_{t-1} + (1-\beta)g_t^2 \\ w_t = w_{t-1} - \alpha \dfrac{g_t}{\sqrt{v_t} + \epsilon} \end{cases}
		\end{equation*}
		\item Adam
		\begin{equation*}
			\begin{cases}m_0 = 0\\ v_0 = 0\\ g_t = \nabla f(w_{t-1}) \\ m_t = \beta_1 m_{t-1} + (1-\beta_1) g_t \\ v_t = \beta_2 v_{t-1} + (1-\beta_2)g_t^2 \\
				\hat{m}_t = \dfrac{m_t}{1 - \beta_1^t} \\ \hat{v}_t = \dfrac{v_t}{1 - \beta_2^t} \\ w_t = w_{t-1} - \alpha \dfrac{\hat{m}_t}{\sqrt{\hat{v}_t} + \epsilon} \end{cases}
		\end{equation*}
		\item Adamax
		\begin{equation*}
			\begin{cases}m_0 = 0\\ u_0 = 0 \\ g_t = \nabla f(w_{t-1}) \\ m_t = \beta_1 m_{t-1} + (1-\beta_1) g_t \\ u_t = \max(\beta_2 u_{t-1}, \lVert g_t \rVert_{\infty}) \\ w_t = w_{t-1} - \alpha \dfrac{m_t}{(1 - \beta_1^t)u_t + \epsilon}\end{cases}
		\end{equation*}
		\item AMSGrad
		\begin{equation*}
			\begin{cases}m_0 = 0\\ v_0 = 0\\ g_t = \nabla f(w_{t-1}) \\ m_t = \beta_1 m_{t-1} + (1-\beta_1) g_t \\ v_t = \beta_2 v_{t-1} + (1-\beta_2)g_t^2 \\ \hat{v}_t = \max(\hat{v}_{t-1}, v_t) \\ w_t = w_{t-1} - \alpha \dfrac{m_t}{\sqrt{\hat{v}_t} + \epsilon} \end{cases}
		\end{equation*}
		\item Nadam
		\begin{equation*}
			\begin{cases}m_0 = 0\\ v_0 = 0\\ g_t = \nabla f(w_{t-1}) \\ m_t = \beta_1 m_{t-1} + (1-\beta_1) g_t \\ v_t = \beta_2 v_{t-1} + (1-\beta_2)g_t^2 \\
				\hat{m}_t = \dfrac{m_t}{1 - \beta_1^t} \\ \hat{v}_t = \dfrac{v_t}{1 - \beta_2^t} \\ w_t = w_{t-1} -  \dfrac{\alpha}{\sqrt{\hat{v}_t} + \epsilon}\Bigg(\beta_1\hat{m}_t + \dfrac{1-\beta_1}{1-\beta_1^t}g_t\Bigg) \end{cases}
		\end{equation*}
		\item AdaBelief
		\begin{equation*}
			\begin{cases}m_0 = 0\\ s_0 = 0\\ g_t = \nabla f(w_{t-1}) \\ m_t = \beta_1 m_{t-1} + (1-\beta_1) g_t \\ s_t = \beta_2 s_{t-1} + (1-\beta_2)(g_t - m_t)^2 + \epsilon\\
				\hat{m}_t = \dfrac{m_t}{1 - \beta_1^t} \\ \hat{s}_t = \dfrac{s_t}{1 - \beta_2^t} \\ w_t = w_{t-1} - \alpha \dfrac{\hat{m}_t}{\sqrt{\hat{s}_t} + \epsilon} \end{cases}
		\end{equation*}
		\item Accelerated Gradient
		\begin{equation*}
			\begin{cases}v_t = w_{t-1} + \dfrac{t - 1}{t + 2} \big(w_{t-1} - w_{t-2} \big) \\
			w_{t+1} = v_t - \alpha \nabla f(v_t) \end{cases}
		\end{equation*}
		\item RAdam
		\begingroup
			\scalebox{0.8}{
				\begin{minipage}{1.2 \linewidth}
					\begin{algorithm}[H]
						\caption{Thuật toán RAdam}
						\hspace*{\algorithmicindent} \textbf{Input:} {độ dài bước $\lbrace \alpha_t \rbrace_{t=1}^{T}$, các hệ số $\beta_1, \beta_2$, $w_0$ khởi tạo, hàm mục tiêu $f(w)$} \\
						\hspace*{\algorithmicindent} \textbf{Output:} {$w$ đã được học}
						\begin{algorithmic}[1]
							\State{$m_0 \leftarrow 0$}
							\State{$v_0 \leftarrow 0$}
							\State{$\rho_{\infty} \leftarrow 2/(1-\beta_2)-1$}
							\For{$t=1$ to $T$}
								\State{$g_t \leftarrow \nabla f(w_{t-1})$}
								\State{$m_t \leftarrow \beta_1 m_{t-1} + (1 - \beta_1)g_t$}
								\State{$v_t \leftarrow \beta_2 v_{t-1} + (1 - \beta_2)g_t^2$}
								\State{$\widehat{m}_t \leftarrow m_t / (1 - \beta_1^t)$}
								\State{$\rho_t \leftarrow \rho_{\infty} - 2t\beta_2^t / (1 - \beta_2^t)$}
								\If {$\rho_t < 4$}
									\State{$\widehat{v}_t \leftarrow \sqrt{v_t / (1 - \beta_2^t)}$}
									\State{$r_t \leftarrow \sqrt{\frac{(\rho_t - 4)(\rho_t - 2)\rho_{\infty}}{(\rho_{\infty} - 4)(\rho_{\infty} - 2)\rho_t}}$}
									\State{$w_t \leftarrow w_{t-1} - \alpha r_t \widehat{m}_t / (\widehat{v}_t + \epsilon)$}
								\Else
									\State{$w_t \leftarrow w_{t-1} - \alpha \widehat{m}_t$}
								\EndIf
							\EndFor
							\State \Return $w_T$
						\end{algorithmic}
					\end{algorithm}
				\end{minipage}
			}
		\endgroup
		\item Avagrad
		\begin{equation*}
			\begin{cases} w_0 \in \mathbb{R}^d \\ m_0 = 0 \\ v_0 = 0 \\
			g_t = \nabla f(w_{t-1}) \\
			m_t = \beta_1 m_{t-1} + (1-\beta_1)g_t \\
			\eta_t = \dfrac{1}{\sqrt{v_{t-1}} + \epsilon} \\
			w_t = w_{t-1} - \alpha \dfrac{\eta_t}{\lVert \eta_t / \sqrt{d} \rVert_2} \odot m_t \\
			v_t = \beta_2 v_{t-1} + (1 - \beta_2) g_t^2\end{cases}
		\end{equation*}
		\item BFGS
		\begin{equation*}
			\begin{cases} \text{Khởi tạo } H_0 \\
			g_t = \nabla f(w_{t-1}) \\
			p_t = H_{t-1} g_t \\
			w_t = w_{t-1} - \alpha p_t \\
			s_t = w_t - w_{t-1} \\
			y_t = \nabla f(w_t) - \nabla f(w_{t-1}) \\
			H_t = \Bigg(I - \dfrac{s_t y_t^T}{y_t^T s_t} \Bigg)H_{t-1}\Bigg( I - \dfrac{y_t s_t^T}{y_t^T s_t} \Bigg) + \dfrac{s_t s_t^T}{y_t^T s_t}\end{cases}
		\end{equation*}
		\framebreak
		\item DFP
		\begin{equation*}
			\begin{cases} \text{Khởi tạo } H_0 \\
			g_t = \nabla f(w_{t-1}) \\
			p_t = H_{t-1} g_t \\
			w_t = w_{t-1} - \alpha p_t \\
			s_t = w_t - w_{t-1} \\
			y_t = \nabla f(w_t) - \nabla f(w_{t-1}) \\
			H_t = H_{t-1} - \dfrac{H_{t-1}y_t y_t^T H_{t-1}}{y_t^T H_{t-1} y_t} + \dfrac{s_t s_t^T}{y_t^T s_t}\end{cases}
		\end{equation*}
	\end{enumerate}
\end{frame}

\end{document}